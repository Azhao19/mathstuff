\documentclass[12pt, reqno]{amsart}

% Import packages
\usepackage{tikz-cd}
\usepackage{amssymb, amsfonts, amsthm, amscd}
\usepackage{mathtools}
\usepackage{manfnt} % literally only for the danger sign
\usepackage{comment} % comment blocks
\usepackage[margin=0.75in]{geometry}
\usepackage{hyperref}
    \hypersetup{colorlinks=true,citecolor=blue,urlcolor =gray,linkcolor=gray,linkbordercolor={1 0 0}}
\usepackage[english]{babel}
\usepackage{verbatim} 
\usepackage{enumerate}
\usepackage{rotating} % Fan likes to rotate things
\usepackage{bookmark} % For final PDF
\usepackage{schemata}
\usepackage{graphics}
\usepackage{graphicx}
\usepackage{xfrac}
\usepackage{tensor} % For typesetting notes from Phong's Spring class
\usepackage{stmaryrd} % TCS font (St. Mary Road symbols)
\usepackage[en-US]{datetime2}

% tikz config
\usetikzlibrary{positioning}
\usetikzlibrary{positioning, matrix, arrows}

% Set equation numbering to finest granularity
\numberwithin{equation}{subsubsection}

% Formal environments
\newtheorem{thm}{Theorem}[subsubsection]
\newtheorem{cor}{Corollary}[thm]
\newtheorem{lemma}[thm]{Lemma}

\newtheorem{prop}[thm]{Proposition}
\newtheorem{defn}[thm]{Definition}
\newtheorem{conj}[thm]{Conjecture}

\newtheorem*{rmk}{Remark}
\newtheorem*{motiv}{Motivation}

% Convenience macros
\newcommand{\A}{\mathbb{A}}
\renewcommand{\O}{\mathcal{O}}
\newcommand{\z}{\mathbb{Z}}
\newcommand{\zp}{\mathbb{Z}_p}
\renewcommand{\c}{\mathbb{C}}
\newcommand{\cp}{\mathbb{C}_p}
\newcommand{\q}{\mathbb{Q}}
\renewcommand{\r}{\mathbb{R}}
\newcommand{\n}{\mathbb{N}}
\newcommand{\subscript}[2]{#1_{\geq #2}}
\newcommand{\subsubscript}[3]{(\subscript{#1}{#2})_{#3}}
\newcommand{\supscript}[2]{#1^{#2}}
\newcommand{\supsupscript}[3]{(\supscript{#1}{#2})^{#3}}
% if you need three su{b, p}scripts you're doing something wrong
\newcommand{\subsupscript}[3]{\subscript{#1}{#2}^{#3}} % e.g., affine line
\newcommand{\e}{\epsilon}
\newcommand{\barred}[1]{\left| #1 \right|}
\newcommand{\braced}[1]{\lbrace #1 \rbrace}
\newcommand{\inparenthesis}[1]{\left( #1 \right)}
\newcommand{\closure}[1]{\overline{#1}}
\renewcommand{\frak}[1]{\mathfrak{#1}}
\newcommand{\curly}[1]{\mathcal{#1}}
\newcommand{\bb}[1]{\mathbb{#1}}
\renewcommand{\phi}{\varphi}
\newcommand{\fff}[1]{\bb{F}_p((#1))} % finite function field in var #1
\newcommand{\namedmorphism}[3]{#1: #2 \rightarrow #3}
\newcommand{\namedsurjection}[3]{#1: #2 \twoheadrightarrow #3}
\newcommand{\namedinjection}[3]{#1: #2 \hookrightarrow #3}
\newcommand{\restrict}[2]{#1|_{#2}}
\newcommand{\completion}[1]{\hat{#1}}
\newcommand{\setdiff}[2]{#1 \setminus #2}
\newcommand{\disjointcup}{\sqcup}
\newcommand{\bigdisjointcup}{\bigsqcup}
\newcommand{\indexedlist}[2]{#1_0, \dots, #1_{#2-1}}
\newcommand{\projection}[3]{\namedmorphism{\pi_{#1}}{#2}{#3}}
\newcommand{\strictsubset}{\subsetneq}
\newcommand{\chaincomplex}[1]{\supscript{#1}{\sbullet}}


% Formatting commands for laziness
\renewcommand{\rm}[1]{\mathrm{#1}}
\renewcommand{\bf}[1]{\textbf{#1}}
\renewcommand{\it}[1]{\textit{#1}}
\newcommand{\annotate}[1]{\marginpar[#1]{#1}} %[on left]{on right}
\newcommand{\spaceinequation}{\:}
\newcommand{\linedelimiter}{\noindent ---------------------------------------------------------------------------------------------------------------------------------------------}

% Arrows for morphisms: #1 is top, #2 is bottom
\newcommand{\rinj}[2]{\xhookrightarrow[#2]{#1}}
\newcommand{\rsurj}[2]{\xtwoheadrightarrow[#2]{#1}}
\newcommand{\linj}[2]{\xhookleftarrow[#2]{#1}}
\newcommand{\lsurj}[2]{\xhooktwoheadleftarrow[#2]{#1}}
\newcommand{\rnormal}[2]{\xrightarrow[#2]{#1}}
\newcommand{\lnormal}[2]{\xleftarrow[#2]{#1}}
\newcommand{\bij}[2]{\xleftrightarrow[#2]{#1}}

% Common expressions
\newcommand{\ellipticcurve}[1]{#1^3 + A#1 + B}
\newcommand{\polynomial}[4]{\sum_{#1 = 0}^{#4}#2_i#3^{#1}} % #1 is index, 
                                                           % #2 is coefficient var, 
                                                           % #3 is polynomial var,
                                                           % #4 is degree
\newcommand{\modulareq}[3]{#1 \equiv #2 \pmod{#3}}         % e.g., x = 1 (mod 5)
\newcommand{\isomorphic}[2]{#1 \simeq #2}

% Common operators
\newcommand{\opone}[2]{\operatorname{#1}_{#2}}                         % helper
\newcommand{\optwo}[3]{\subsupscript{\operatorname{#1}}{#2}{#3}}       % helper
\newcommand{\im}[1]{\opone{Im}{#1}}
\newcommand{\ord}[1]{\opone{ord}{#1}}
\renewcommand{\gcd}[2]{\operatorname{gcd}(#1, #2)} % \rm{gcd} -> gcd if static scope
\renewcommand{\dim}[1]{\opone{dim}{#1}}
\renewcommand{\deg}[1]{\opone{deg}{#1}}
\newcommand{\length}[2]{\operatorname{length}_{#1}{#2}}
\newcommand{\vsdim}[2]{\operatorname{dim}_{#1}{#2}}
\newcommand{\trdeg}[2]{\operatorname{tr.deg}_{#1}{#2}}
\newcommand{\Cl}[1]{\opone{Cl}{#1}}                          % e.g., Cl_Q = { 1 }
\newcommand{\Gal}[1]{\opone{\curly{G}}{#1}}
\newcommand{\Spec}[1]{\opone{Spec}{#1}}
\newcommand{\Proj}[1]{\opone{Proj}{#1}}
\newcommand{\Frob}[1]{\opone{Frob}{#1}}
\newcommand{\Aut}[2]{\optwo{Aut}{#1}{#2}} % sub == over what thing, sup == obj
\newcommand{\End}[2]{\optwo{End}{#1}(#2)} 
\newcommand{\Ind}[2]{\optwo{Ind}{#1}{#2}}
\newcommand{\Res}[2]{\optwo{Res}{#1}{#2}}
\newcommand{\Tor}[4]{\operatorname{Tor}_{#1}^{#2}(#3, #4)}
\newcommand{\Ext}[4]{\operatorname{Ext}_{#1}^{#2}(#3, #4)}
\newcommand{\ithcohomology}[4]{H^{#1}(#2, #3)_{#4}}        % #4 == type of coh
\newcommand{\ithhomology}[3]{H_{#1}(#2, #3)}
\newcommand{\quotient}[2]{#1/#2}                           % () added manually
\newcommand{\nthofring}[2]{#1/#2#1}                        % e.g., Z/nZ
\newcommand{\ntorsion}[2]{#1[#2]}                          % e.g., Sel[\varphi]
\newcommand{\unitsofring}[2]{\supscript{#1}{#2}}           % e.g., R^*
\newcommand{\abelianization}[1]{\supscript{#1}{\rm{ab}}} % K^{ab}
\newcommand{\adjoin}[2]{#1[#2]}
\newcommand{\laurentring}[2]{#1\llbracket #2 \rrbracket}
\newcommand{\fiberproduct}[3]{#2 \cross_{#1} #3}
\newcommand{\directsum}[2]{\oplus_{#1} #2}                 % for sets
\newcommand{\directproduct}[2]{\prod_{#1}{#2}}             % for sets
\newcommand{\tensorproduct}{\otimes}


% Other symbols
\newcommand{\warning}{\dbend}
\newcommand{\existsuniquely}{\exists!}
\newcommand{\doesntexist}{\nexists}
\newcommand{\begincomment}{\comment} % \endcomment is already defined in comment

% Common tikz Diagrams
% You're going to have to use the LaTeX arrow mnemonics
    % color: color
    % hook: inj
    % two heads: surj
    % dashed, squiggly shapes, dash, equal
% Each arrow arg will be a 4-tuple (color, shape, design, label)
% N.B.: label must be of the proper syntax, with ' indicating left or down
% positions.

% Commutative Squares
\newcommand{\commsquare}[8]{\[ % first 4 args are vertices, last 4 arrows
                               % interpret as two clockwise lists
\begin{tikzcd}
        #1 && #2 \\
	             \\
	    #4 && $3
	\arrow[from=1-1, to=1-3, #5] % each of the arrow args
	\arrow[from=1-3, to=3-3, #6] % is a 4-tuple
    \arrow[from=1-1, to=3-1, #7] % (color, hook/two heads, dashed/./equal, label)
	\arrow[from=3-1, to=3-3, #8] % if unused attr, leave blank
\end{tikzcd}\]}

% Triangular Universal Property (Isoceles)
\newcommand{\triangluaruniversalproperty}[5] {\[ % 3 vert's clockwise from left
                                                 % name of top, then left arrow
\begin{tikzcd}
    #1 && #2 \\
	& #3
    \arrow[from=1-1, to=1-3, #4]
	\arrow["{\exists!}", dashed, from=1-3, to=2-2]
    \arrow[from=1-1, to=2-2, #5]
\end{tikzcd}
\]}

% Natural Transformation
\newcommand{\naturaltransformation}[5]{\[ % src then dst cat, functors, nat trans
\begin{tikzcd}
    #1 && #2
	\arrow[""{name=1, anchor=center, inner sep=0}, "#3", curve={height=-12pt}, from=1-1, to=1-3]
	\arrow[""{name=0, anchor=center, inner sep=0}, "#4"', curve={height=12pt}, from=1-1, to=1-3]
    \arrow[#5, shorten <=3pt, shorten >=3pt, Rightarrow, from=1, to=0]
\end{tikzcd}
\]}

% Short Exact Sequence
\newcommand{\shortexactsequence}[5]{\[ % 3 objects and 2 morphisms
\begin{tikzcd}
	0 & #1 & #2 & #3 & 0 \\
	\arrow[from=1-1, to=1-2]
    \arrow[from=1-2, to=1-3, #4]
    \arrow[from=1-3, to=1-4, #5]
	\arrow[from=1-4, to=1-5]
\end{tikzcd}
\]}

\newcommand{\sesinternal}[5] {\[
	0 & #1 & #2 & #3 & 0 \\
	\arrow[from=1-1, to=1-2]
    \arrow[from=1-2, to=1-3, #4]
    \arrow[from=1-3, to=1-4, #5]
	\arrow[from=1-4, to=1-5]
\]}

% Long Exact Sequences (tex stack exchange post 3892)
\newcommand{\longexactsequencewithcoefficients}[4] {\[ % 1st argument is coeff
                                                       % 3 ses objects

\begin{tikzpicture}[descr/.style={fill=white,inner sep=1.5pt}]
        \matrix (m) [
            matrix of math nodes,
            row sep=1em,
            column sep=2.5em,
            text height=1.5ex, text depth=0.25ex
        ]
        { \dots & \ithcohomology{n-1}{#2}{#1} & \ithcohomology{n-1}{#3}{#1}
                & \ithcohomology{n-1}{#4}{#1} \\
            &  \ithcohomology{n}{#2}{#1} & \ithcohomology{n}{#3}{#1} &
               \ithcohomology{n}{#4}{#1} \\
            &  \ithcohomology{n+1}{#2}{#1} & \dots & \mbox{}
        };

        \path[overlay,->, font=\scriptsize,>=latex]
        (m-1-1) edge (m-1-2)
        (m-1-2) edge (m-1-3)
        (m-1-3) edge (m-1-4)
        (m-1-4) edge[out=355,in=175] node[descr,yshift=0.3ex] {$\delta_{n-1}$} (m-2-2)
        (m-2-2) edge (m-2-3)
        (m-2-3) edge (m-2-4)
        (m-2-4) edge[out=355,in=175] node[descr,yshift=0.3ex] {$\delta_n$} (m-3-2)
\end{tikzpicture}
\]}

\newcommand{\longexactsequence}[3] {\[

\begin{tikzpicture}[descr/.style={fill=white,inner sep=1.5pt}]
        \matrix (m) [
            matrix of math nodes,
            row sep=1em,
            column sep=2.5em,
            text height=1.5ex, text depth=0.25ex
        ]
        { \dots & H^{n-1}(#1) & H^{n-1}(#2) & H^{n-1}(#3) \\
                &  H^{n}(#1) & H^{n}(#2) & H^{n}(#3) \\
                &  H^{n+1}(#1) & \dots & \mbox{} \\
        };

        \path[overlay,->, font=\scriptsize,>=latex]
        (m-1-1) edge (m-1-2)
        (m-1-2) edge (m-1-3)
        (m-1-3) edge (m-1-4)
        (m-1-4) edge[out=355,in=175] node[descr,yshift=0.3ex] {$\delta_{n-1}$} (m-2-2)
        (m-2-2) edge (m-2-3)
        (m-2-3) edge (m-2-4)
        (m-2-4) edge[out=355,in=175] node[descr,yshift=0.3ex] {$\delta_n$} (m-3-2)
\end{tikzpicture}
\]}

% A 1 by Short Exact Sequence Grid
\newcommand{\linkedshortexactsequences}[9]{\[ % arg 1: upper ses w/o tikz env
                                             % 5 args for lower ses
                                             % 3 args for vert arrows
                    
\begin{tikzcd}
    #1
    \\
    0 & #2 & #3 & #4 & 0
	\arrow[from=3-1, to=3-2]
    \arrow[from=3-2, to=3-3, #5]
    \arrow[from=3-3, to=3-4, #6]
	\arrow[from=3-4, to=3-5]
    \arrow[from=1-2, to=3-2, #7]
    \arrow[from=1-3, to=3-3, #8]
    \arrow[from=1-4, to=3-4, #9]
\end{tikzcd}
\]}


% Fiber Product Diagram
    % We give all arrows names except for the dashed one to avoid
    % having to do any arrow formatting.
\newcommand{\universalfiberproduct}[8]{\[ % arg order:
                                           % 3 vertices of sq from top right
                                           % name of right map
                                           % " bottom map
                                           % " contending object
                                           % " arrow top right, bottom left
                                           % 5th and 8th should have ' on labels
\begin{tikzcd}                                       
    #6 \\
    & \fiberproduct{#2}{#3}{#1} && #1 \\
	\\
	& #3 && #2
	\arrow["{\pi_1}", from=2-2, to=2-4]
    \arrow[from=2-4, to=4-4, #4]
	\arrow["{\pi_2}"', from=2-2, to=4-2]
    \arrow[from=4-2, to=4-4, #5]
    \arrow[curve={height=-12pt}, from=1-1, to=2-4, #7]
    \arrow[curve={height=12pt}, from=1-1, to=4-2, #8]
	\arrow["{\exists!}", dashed, from=1-1, to=2-2]
\end{tikzcd}
\]}

% Pullback Diagram
\newcommand{\pullbackdiagram}[6]{\[ % 4 vertices clockwise, then
                                    % names of right and bottom map
                                    % 6th should have ' on label
\begin{tikzcd}
    #1 && #2 \\
	\\
    #3 && #4
	\arrow["\lrcorner"{anchor=center, pos=0.125}, draw=none, from=1-1, to=3-3]
    \arrow["{\pi_1}", from=1-1, to=3-1]
    \arrow[from=1-1, to=1-3, #5]
    \arrow[from=1-3, to=3-3, #6]
    \arrow["{\pi_2}"', from=3-1, to=3-3]
\end{tikzcd}
\]}

% Pushout Diagram
\newcommand{\pushoutdiagram}[6]{\[ % 4 vertices clockwise, then
                                   % names of right and bottom map
                                   % 6th should have ' on label
\begin{tikzcd}
    #1 && #2 \\
	\\
    #3 && #4
	\arrow["\ulcorner"{anchor=center, pos=0.125}, draw=none, from=1-1, to=3-3]
    \arrow["{\pi_1}", from=1-1, to=3-1]
    \arrow[from=1-1, to=1-3, #5]
    \arrow[from=1-3, to=3-3, #6]
    \arrow["{\pi_2}"', from=3-1, to=3-3]
\end{tikzcd}
\]}

\author{Alan Zhao}
\date{\today}
\email{asz2115@columbia.edu}
